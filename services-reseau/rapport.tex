\documentclass[12pt, a4paper]{article}
\usepackage[francais]{babel}
\usepackage{caption}
\usepackage{graphicx}
\usepackage[T1]{fontenc}
\usepackage{listings}
\usepackage{geometry}
\usepackage[colorlinks=true,linkcolor=black,anchorcolor=black,citecolor=black,filecolor=black,menucolor=black,runcolor=black,urlcolor=black]{hyperref}

% \usepackage{mathpazo} --> Police à utiliser lors de rapports plus sérieux

\usepackage{fancyhdr}
\pagestyle{fancy}
\lhead{}
\rhead{}
\chead{}
\rfoot{\thepage}
\lfoot{Martin Baumgaertner}
\cfoot{}

\renewcommand{\headrulewidth}{0.4pt}
\renewcommand{\footrulewidth}{0.4pt}

\begin{document}
\begin{titlepage}
	\newcommand{\HRule}{\rule{\linewidth}{0.5mm}} 
	\center 
	\textsc{\LARGE iut de colmar}\\[6.5cm] 
	\textsc{\Large saé - concevoir un réseau multi-sites}\\[0.5cm] 
	\textsc{\large Année 2022-23}\\[0.5cm]
	\HRule\\[0.75cm]
	{\huge\bfseries Partie services réseaux}\\[0.4cm]
	\HRule\\[1.5cm]
	\textsc{\large martin baumgaertner}\\[6.5cm] 

	\vfill\vfill\vfill
	{\large\today} 
	\vfill
\end{titlepage}
\newpage
\tableofcontents
\newpage
\section{Introduction}
Le but principale de cette SAÉ fût bien sûr de développer un réseau informatique
qui s'étend sur plusieurs sites. Pour cela, mes camarades ont utilisés des protocoles
de routage propore à leurs besoins. Pour ma part, j'ai choisi de mettre en place
les services réseaux. Donc bien entendu, les protocoles et technologies utilisés
ont été différentes. Je vais donc vous expliquer à travers ce rapport, comment
j'ai mis en place les différents services réseaux nécessaires au projet,
à savoir ; les DNS, le serveur DHCP, le serveur WEB et le serveur mail.


\section{les DNS}
    \subsection{Explication}
    Pour la mise en place des DNS, j'ai utilisé le logiciel Bind9 
    le logiciel libre Bind9, présent exclusivement sur Linux. J'ai fait 
    le choix de ce logiciel car c'était celui que nous utilisions pendant
    les cours de M. BINDEL en début d'année, mais également celui que nous 
    utilisions en TP l'an dernier avec Mme LACROIX. 


    \subsection{Le DNS primaire}
    Pour la mise en place du DNS primaire, j'ai crée un contenur LXC sous
    ubuntu. J'y ai donc installé Bind9 avec la commande suivante : 
    \texttt{sudo apt install bind9}. Puis, j'ai configuré les fichiers
    de configuration de bind9 qui se trouvent tous dans le répertoire 
    \texttt{/etc/bind/}. Pour commencer, il faut déclarer une zone. 
    Pour ce faire, il faut éditer le fichier named.conf.local. Voici la
    configuration que j'ai mis dans ce fichier :

    \subsection{Le DNS secondaire}


\section{le DHCP}
    \subsection{Explication du DHCP}

    \section{Mise en place du DHCP}


\section{Le serveur WEB}


\section{Le serveur mail}


\end{document}